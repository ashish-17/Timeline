Following are the deliverables for this stage:

\paragraph{General Description}
This project is like a minimalistic search engine which gives collated
information and events about the query company or person in form of a timeline.

\paragraph{User Type 1}
Job-hunters who want to know background of a company or a person.
\begin{itemize}
\item User Interaction Modes: Keyword based search for people exploring the web.
\item Real World Scenarios:
  \begin{itemize}
  \item Scenario 1 Description: A Person trying to get insight into a company
    like "Microsoft".
  \item System Data Input for Scenario 1: "Microsoft".
  \item Input Data Types for Scenario 1: String (Spaces allowed).
  \item System Data Output for Scenario 1: All data related to "Microsoft"
    placed on a timeline.
  \item Output Data Types for Scenario 1: Date of event, text corresponding to
    the event and URLs of the sources of information.
  \item Scenario 2 Description: A student trying to know about some famous
    person like "Alan Turing".
  \item System Data Input for Scenario 2: "Alan Turing".
  \item Input Data Types for Scenario 2: String (Spaces allowed).
  \item System Data Output for Scenario 2: All the event's data related to
    "Alan Turing" placed on a timeline.
  \item Output Data Types for Scenario 2: Date of event, text corresponding to
    the event and URLs of the sources of information.
  \end{itemize}
\end{itemize}

\paragraph{User Type 2}
Researchers or students who want to look for techniques or patents created by a
person.
\begin{itemize}
\item User Interaction Modes: Keyword based search for people exploring the web.
\item Real World Scenarios:
  \begin{itemize}
  \item Scenario 1 Description: A researcher who collects technique inventions
    by a company like "Google".
  \item System Data Input for Scenario 1: "Google".
  \item Input Data Types for Scenario 1: String (Spaces allowed).
  \item System Data Output for Scenario 1: All research data related to
    "Google" placed on a timeline filtered by data category.
  \item Output Data Types for Scenario 1: Date of event, text corresponding to
    the event and URLs of the sources of information.
  \item Scenario 2 Description: A student trying to collect papers by some
    researchers like "James Abello".
  \item System Data Input for Scenario 2: "James Abello".
  \item Input Data Types for Scenario 2: String (Spaces allowed).
  \item System Data Output for Scenario 2: List of papers published by "James
    Abello" placed on a timeline.
  \item Output Data Types for Scenario 2: Date of event, text corresponding to
    the event and URLs of the sources of information.
  \end{itemize}
\end{itemize}

\paragraph{User Type 3}
Analysts who would like to fetch financial statistics data of a company
\begin{itemize}
\item User Interaction Modes: RESTful API access for researchers interested in
  our event data.
\item Real World Scenarios: 
  \begin{itemize} 
  \item Scenario 1 Description: Researchers at a company 'Bloomberg' trying to
    study their stock fluctuations over the past months and want to map the
    stock data to the event's data.
  \item System Data Input for Scenario 1: A RESTful request of GET type.
  \item Input Data Types for Scenario 1: String (URL).
  \item System Data Output for Scenario 1: Date of event, text corresponding to
    the event and URLs of the sources of information.
  \item Output Data Types for Scenario 1: JSON.
  \item Scenario 2 Description: Students in Rutgers University want to study
    the marketing and publicity patterns for multiple organizations.
  \item System Data Input for Scenario 2: Multiple REST requests, one for each
    organization.
  \item Input Data Types for Scenario 2: String (URL).
  \item System Data Output for Scenario 2: All the event's data consisting of
    date of event, text corresponding to the event and URLs of the sources of
    information.
  \item Output Data Types for Scenario 2: JSON
  \end{itemize}
\end{itemize}

\paragraph{Project Timeline and Division of Labor}
The project has three main components - Automated web scrapper/crawler, web
application with MongoDB backend and timeline visualization, and Machine
learning techniques used on Hadoop to extract events of interest. Each of the
team member will be responsible for completion of one of the above tasks,
including work related to development, testing and documentation. We estimate
the project completion in about 6 weeks. 

\begin{itemize}
\item Week 1: By the end of week 1 we expect to have identified all the seeding
  sources for our application and built a working multi-threaded web-scraper.
\item Week 2: After we have tested it's working on localhost, 1 person in team
  will be responsible for deploying Hadoop in fully-distributed mode which will
  be integrated into web application. In parallel other members will start
  building the web application and exploring the machine learning on event
  extraction and topic models. At the end of week 2 we expect to have bunch of
  data crawled by web-scraper and a functional web application front end with
  Hadoop platform.
\item Week 3: In week 3 we will link our web application with a MongoDB backend
  which will be loaded with some dummy data to make the web application fully
  operational to test its functionality. Meanwhile, we will also be working on
  cleaning the data and testing some machine learning algorithm to extract event
  data from it.
\item Week 4: We will implement some effective models and integrate them into
  web application. At the end of week 4 we should be able to extract useful
  event information and save it into MongoDB to be available for query from web
  application.
\item Week 5: In week 5 we will work on testing and making the application
  stable.
\item Week 6: In week 6 we will work on making the application stable. At the
  end of week 6 we should have a stable working system with final project
  report and presentation.
\end{itemize}
