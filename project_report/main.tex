\documentclass[10pt,conference]{IEEEtran}

\ifCLASSINFOpdf
	\usepackage[pdftex]{graphicx}
	\graphicspath{{./figures/}}
	\DeclareGraphicsExtensions{.pdf,.jpeg,.png}
\else
	\usepackage[dvips]{graphicx}
	%\graphicspath{{./figures/}}
	\DeclareGraphicsExtensions{.eps}
\fi

\usepackage[cmex10]{amsmath}
\usepackage[tight,footnotesize]{subfigure}
\usepackage{xcolor}
\usepackage[lined,ruled]{algorithm2e}
\usepackage[latin1]{inputenc}
\usepackage{tikz}
\usetikzlibrary{shapes}
\usetikzlibrary{arrows}

\newtheorem{property}{Property}
\newtheorem{proposition}{Proposition}
\newtheorem{theorem}{Theorem}
\newtheorem{conjecture}{Conjecture}
\newtheorem{question}{Question}
\newtheorem{definition}{Definition}
\newtheorem{corollary}{Corollary}

\makeatletter
\pgfdeclareshape{datastore}{%
\inheritsavedanchors[from=rectangle]
\inheritanchorborder[from=rectangle]
\inheritanchor[from=rectangle]{center}
\inheritanchor[from=rectangle]{base}
\inheritanchor[from=rectangle]{north}
\inheritanchor[from=rectangle]{north east}
\inheritanchor[from=rectangle]{east}
\inheritanchor[from=rectangle]{south east}
\inheritanchor[from=rectangle]{south}
\inheritanchor[from=rectangle]{south west}
\inheritanchor[from=rectangle]{west}
\inheritanchor[from=rectangle]{north west}
\backgroundpath{%
  %  Store lower right in xa/ya and upper right in xb/yb
  \southwest \pgf@xa=\pgf@x \pgf@ya=\pgf@y
  \northeast \pgf@xb=\pgf@x \pgf@yb=\pgf@y
  \pgfpathmoveto{\pgfpoint{\pgf@xa}{\pgf@ya}}
  \pgfpathlineto{\pgfpoint{\pgf@xb}{\pgf@ya}}
  \pgfpathmoveto{\pgfpoint{\pgf@xa}{\pgf@yb}}
  \pgfpathlineto{\pgfpoint{\pgf@xb}{\pgf@yb}}
  }
}
\makeatother

\newcommand{\riham}[1]{{\color{red}{#1}}}
\newcommand{\james}[1]{{\color{blue}{#1}}}
\newcommand{\yx}[1]{{\color{red}{#1}}}


\begin{document}

\title{A Hadoop-driven Data Analysis System}

\author{%
\IEEEauthorblockN{Ashish Jindal}
\IEEEauthorblockA{Rutgers University\\
Piscataway, NJ, USA\\
Email: ashish.jindal@rutgers.edu}
\and
\IEEEauthorblockN{Yikun Xian}
\IEEEauthorblockA{Rutgers University\\
Piscataway, NJ, USA\\
Email: siriusxyk@gmail.com}
\and
\IEEEauthorblockN{Sanjivi Muttena}
\IEEEauthorblockA{Rutgers University\\
Piscataway, NJ, USA\\
Email: sm1727@scarletmail.rutgers.edu}
}

\maketitle

\begin{abstract}
\textnormal{%
The complexity of modern analytics needs is outstripping the available
computing power of legacy systems. Distributed system like Hadoop compliments
this requirement for storing and analyzing huge sets of information by
providing a platform for parallel processing of large data sets stored over
multiple machines. This project aims to setup a Hadoop infrastructure and
demonstrate its power for big data analysis. Some machine learning models will
also be deployed in this system so that developers can directly call
corresponding APIs to execute training and testing models.
}
\end{abstract}

\begin{IEEEkeywords}
  \textnormal{Hadoop-driven, Data Analysis, Infrastructure}
\end{IEEEkeywords}

\IEEEpeerreviewmaketitle

\section{Project Description}\label{sec:project-description}

Apache Hadoop is an open-source software framework written in Java for
distributed storage and distributed processing of very large data sets on
computer clusters built from commodity hardware. Hadoop stores data as it comes
in - structured or unstructured - saving time on configuring data for
relational databases. In our project we will store a large dataset in Hadoop
file system and use it for analysis of events from various sources like news,
twitter etc. Our project falls in the category \textit{Scalable Algorithms
Infrastructure}. The main stumbling points for this project are identifying a
large dataset for Hadoop, configuring Hadoop for multiple clusters and building
a data analytics system over Hadoop.

The project has four stages: requirement gathering, design, infrastructure
implementation, and user interface.

\subsection{Stage1 - The Requirement Gathering Stage}
\label{sec:requirement-gathering-stage}
Following are the deliverables for this stage:

\paragraph{General Description}
This project is like a minimalistic search engine which gives collated
information and events about the query company or person in form of a timeline.

\paragraph{User Type 1}
Job-hunters who want to know background of a company or a person.
\begin{itemize}
\item User Interaction Modes: Keyword based search for people exploring the web.
\item Real World Scenarios:
  \begin{itemize}
  \item Scenario 1 Description: A Person trying to get insight into a company
    like "Microsoft".
  \item System Data Input for Scenario 1: "Microsoft".
  \item Input Data Types for Scenario 1: String (Spaces allowed).
  \item System Data Output for Scenario 1: All data related to "Microsoft"
    placed on a timeline.
  \item Output Data Types for Scenario 1: Date of event, text corresponding to
    the event and URLs of the sources of information.
  \item Scenario 2 Description: A student trying to know about some famous
    person like "Alan Turing".
  \item System Data Input for Scenario 2: "Alan Turing".
  \item Input Data Types for Scenario 2: String (Spaces allowed).
  \item System Data Output for Scenario 2: All the event's data related to
    "Alan Turing" placed on a timeline.
  \item Output Data Types for Scenario 2: Date of event, text corresponding to
    the event and URLs of the sources of information.
  \end{itemize}
\end{itemize}

\paragraph{User Type 2}
Researchers or students who want to look for techniques or patents created by a
person.
\begin{itemize}
\item User Interaction Modes: Keyword based search for people exploring the web.
\item Real World Scenarios:
  \begin{itemize}
  \item Scenario 1 Description: A researcher who collects technique inventions
    by a company like "Google".
  \item System Data Input for Scenario 1: "Google".
  \item Input Data Types for Scenario 1: String (Spaces allowed).
  \item System Data Output for Scenario 1: All research data related to
    "Google" placed on a timeline filtered by data category.
  \item Output Data Types for Scenario 1: Date of event, text corresponding to
    the event and URLs of the sources of information.
  \item Scenario 2 Description: A student trying to collect papers by some
    researchers like "James Abello".
  \item System Data Input for Scenario 2: "James Abello".
  \item Input Data Types for Scenario 2: String (Spaces allowed).
  \item System Data Output for Scenario 2: List of papers published by "James
    Abello" placed on a timeline.
  \item Output Data Types for Scenario 2: Date of event, text corresponding to
    the event and URLs of the sources of information.
  \end{itemize}
\end{itemize}

\paragraph{User Type 3}
Analysts who would like to fetch financial statistics data of a company
\begin{itemize}
\item User Interaction Modes: RESTful API access for researchers interested in
  our event data.
\item Real World Scenarios: 
  \begin{itemize} 
  \item Scenario 1 Description: Researchers at a company 'Bloomberg' trying to
    study their stock fluctuations over the past months and want to map the
    stock data to the event's data.
  \item System Data Input for Scenario 1: A RESTful request of GET type.
  \item Input Data Types for Scenario 1: String (URL).
  \item System Data Output for Scenario 1: Date of event, text corresponding to
    the event and URLs of the sources of information.
  \item Output Data Types for Scenario 1: JSON.
  \item Scenario 2 Description: Students in Rutgers University want to study
    the marketing and publicity patterns for multiple organizations.
  \item System Data Input for Scenario 2: Multiple REST requests, one for each
    organization.
  \item Input Data Types for Scenario 2: String (URL).
  \item System Data Output for Scenario 2: All the event's data consisting of
    date of event, text corresponding to the event and URLs of the sources of
    information.
  \item Output Data Types for Scenario 2: JSON
  \end{itemize}
\end{itemize}

\paragraph{Project Timeline and Division of Labor}
The project has three main components - Automated web scrapper/crawler, web
application with MongoDB backend and timeline visualization, and Machine
learning techniques used on Hadoop to extract events of interest. Each of the
team member will be responsible for completion of one of the above tasks,
including work related to development, testing and documentation. We estimate
the project completion in about 6 weeks. 

\begin{itemize}
\item Week 1: By the end of week 1 we expect to have identified all the seeding
  sources for our application and built a working multi-threaded web-scraper.
\item Week 2: After we have tested it's working on localhost, 1 person in team
  will be responsible for deploying Hadoop in fully-distributed mode which will
  be integrated into web application. In parallel other members will start
  building the web application and exploring the machine learning on event
  extraction and topic models. At the end of week 2 we expect to have bunch of
  data crawled by web-scraper and a functional web application front end with
  Hadoop platform.
\item Week 3: In week 3 we will link our web application with a MongoDB backend
  which will be loaded with some dummy data to make the web application fully
  operational to test its functionality. Meanwhile, we will also be working on
  cleaning the data and testing some machine learning algorithm to extract event
  data from it.
\item Week 4: We will implement some effective models and integrate them into
  web application. At the end of week 4 we should be able to extract useful
  event information and save it into MongoDB to be available for query from web
  application.
\item Week 5: In week 5 we will work on testing and making the application
  stable.
\item Week 6: In week 6 we will work on making the application stable. At the
  end of week 6 we should have a stable working system with final project
  report and presentation.
\end{itemize}


%\subsection{Stage2 - The Design Stage. }\label{sec: 2:The Design Stage.}
%%%%%%%%%%%%%%%%%%%%%%%%%%%%%%%%%%%%%%%%%%%%%%%%%%%%%%%%%%%%%%%%%%%%%%%%%%%%%%%%%%%%%%%%%%%%%%%%%%%%%%%%%%
%\textnormal{
Transform the project requirements into a system flow diagram, specifyng the different algorithms, data types and structures required for processing and their associated operations.  
The deliverables for this stage include the system flow diagram containing a graphical representation and  textual descriptions of the corresponding data trasnformations, high level pseudo code of the overall system operation, and overall system time and space complexity.}

%\begin{itemize} 
%\item{ }
%A brief textual description of the overall flow diagram (along with its functional operation in the different user scenarios described in the first stage of the project).
%\item{ }
%A specification of each algorithm and associated data structures together with its entities, attributes, and operations ( include an English description of how they relate to your user scenario(s)).

%\end{itemize}
Please insert your deliverables for Stage2 as follows:
\begin{itemize} 
\item{  Short Textual Project Description. }
Please insert here the flow diagram textual description here together with its overall time and space complexity.
\item{ Flow Diagram. }
Please insert your system Flow Diagram here.
\item{ High Level Pseudo Code System Description. }
Please insert high level pseudo-code describing the major system modules as per your flow diagram.
\item{Algorithms and  Data Structures. }
Please insert a brief description of each major Algorithm and its associated data structures here.
\end{itemize}

\begin{itemize} 
\item{  Flow Diagram Major Constraints.}
Please insert here the integrity constraints:
\begin{itemize} 
\item{ Integrity Constraint. }
Please insert the first integrity constraint in here together with its description and justification. }
\end{itemize}
Please repeat the pattern for each integrity constraint.
\end{itemize}
}


%\subsection{Stage3 - The Implementation Stage. }\label{sec: 3 The Implementation Stage.}
%%%%%%%%%%%%%%%%%%%%%%%%%%%%%%%%%%%%%%%%%%%%%%%%%%%%%%%%%%%%%%%%%%%%%%%%%%%%%%%%%%%%%%%%%
%\textnormal{
Specify the language and programming environemnt you used for your implementation.
%Building the corresponding relational tables, according to the proposed ER model described in the previous phase %enforcing the different integrity constraints.  
The deliverables for this stage include the following items:
\begin{itemize} 
\item{}
Sample small data snippet. 
%The SQL tables that represent the ER project model, along with at least 3-5 rows of concrete data per table.
\item{}
Sample small output
%The normalization steps for each table, along with explanations/justifications of each normalization step.
\item{}
Working code
%The SQL table after the normalization steps (showing all table attributes).
\item{}
Demo and sample findings
%The SQL statements used to create the SQL tables, including the required triggers as well as the integrity constraints. At %least 2 triggers and 2 of each of the following constraint types have to exist in the project tables overall: 
\begin{itemize} 
\item{}
	Data size: In terms of  RAM size;  Disk Resident?; Streaming ?;  
\item{}
	List the most interestng findings in the data if it is a Data Exploration Project. For other project types consult with your project supervisor what the corresponding outcomes shall be. Concentrate on demonstrating the Usefuness and Novelty of your application.
%Whether some users will be denied access and/or updates to some data according to their roles (for example: student1 %can not access other students' ' grades, so a violation error pops up upon that action. Another example: a sales person %can see an item price, but can not change it, since only a manger can, also a violation error pops up upon that update %attempt).
\end{itemize}
\end{itemize}
}


%\subsection{Stage4 -	User Interface. }\label{sec: 4. User Interface.}
%%%%%%%%%%%%%%%%%%%%%%%%%%%%%%%%%%%%%%%%%%%%%%%%%%%%%%%%%%%%%%%%%%%%%%%%%%%%%%%%%%%%%%%%%%%%%%%%%%%%%%%%%%
%\textnormal{
Describe a User Interface (UI) to your application along with the related information that will be shown on each interface view (How users will query or navigate the data and view the query or navigation results). The emphasis should be placed on the process a user needs to follow in order to meet a particular information need in a user-friendly manner.
The deliverables for this stage include the following items :
}
\begin{itemize} 
\item{The modes of user interaction with the data (text queries, mouse hovering, and/or mouse clicks ?).} 
\item{The error messages that will pop-up when users access and/or updates are denied   }
\item{The information messages or results that wil pop-up in response to user interface events. }
	
\item{ The error messages in response to data range constraints violations.}
	
\item{ The interface mechanisms that activate different views in order to facilitate data accesses, according to users'  needs. }
	
\item{Each view created must be justified. Any triggers built upon those views should be explained and justified as well. At least one project view should be created with a justification for its use. }	
\end{itemize}

Please insert your deliverables for Stage4 as follows:
\begin{itemize} 
\item{The initial statement to activate your application with the corresponding initial UI screenshot}
	
\item{Two different  sample navigation user paths through the data exemplifying the different modes of interaction and the corresponding screenshots. }
\item{}
	The error messages popping-up when users access and/or updates are denied (along with explanations and examples):
	\begin{itemize} 
	\item{The error message: }
	\item{The error message explanation (upon which violation it takes place): }
	Please insert the error message explanation in here.
	\item{The error message example according to user(s) scenario(s): }
	Please insert the error message example in here.
	 \end{itemize}
\item{}
	The information messages or results that pop-up in response to user interface events.
	\begin{itemize} 
	\item{The information message: }
	Please insert the error message in here.
	\item{The information message explanation and the corresponding event trigger }
	\item{The error message example in response to data range constraints and the coresponding user's scenario }
	Please insert the error message example in here.
	 \end{itemize}
\item{}
	The  interface mechanisms that activate different views.
	\begin{itemize} 
	\item{The interface mechanism: }
	Please insert the interface mechanism here.
	 \end{itemize}

}


%\section{Project Highlights.}\label{sec:7. Project Highlights.}
%%%%%%%%%%%%%%%%%%%%%%%%%%%%%%%%%%%%%%%%%%%%%%%%%%%%%%%%%%%%%%%%%%%%%%%%%%%%%%%%%%%%%%%%%%%%%%%%%%%%%%%%%%
%\textnormal{
\begin{itemize} 
\item{}
Only working applications will be acceptable at project completion. A running demo shoul be presented to your project advisor at a date to be specified after the second midterm. A version of your application shall be installed in a machine to be specifed later during the semester. Your final submissiom package will also include a final LaTeX report modeled after this document, as well as a Power Point Presentation.
\item{}
The presentation (7 to 8 minutes) should include at least the following items (The order of the slides is important):
\begin{enumerate}
\item{}
Title: Project Names (authors and affiliations)
\item{}
Project Goal
\item{}
Outline of the presentation
\item{}
Description
\item{}
Pictures are essential. Please include Interface snapshots exemplyfing tthe different modes of users's interaction.
\item{}
Project Stumbling Blocks
\item{}
Data collection, Flow Diagram, Integrity Constraints
\item{}
Sample Findings
\item{}
Future Extensions
\item{}
Acknowledgements
\item{}
References and Resources used(libraries, languages, web resources)
\item{}
Demo(3 minutes)
\end{enumerate}
Please follow the sample presentation mock up that is posted on Sakai.
\item{}
By Dec 1 your group should have completed the final submission. This includes a presentation (7 to 8 minutes) to your project advisor as well as a convincing  demo of your project functionalities (3 minutes): every group member should attend the demo (and presentation) indicating clearly  and specifically his/her contribution to the project.  This wil allow us to evaluate all students in a consistent and fair manner.
\item{}
Thank you, and best of luck!
\end{itemize}
}


\bibliographystyle{IEEEtran}
\bibliography{main}
\nocite{*}

\end{document}
